\documentclass{beamer}
\usetheme{Madrid}
\usecolortheme{default}

\usepackage{amsmath}
\usepackage{amssymb}
\usepackage{booktabs}
\usepackage{tikz}
\usepackage[table]{xcolor}

\title{Verifying Quantum Error Correction Codes with SAT Solvers}
\subtitle{Finding Bugs in Surface Code Implementations}
\author{Pengyu Liu, Mengdi Wu}
\date{December 1, 2025}

\begin{document}

\frame{\titlepage}

\begin{frame}{Outline}
  \tableofcontents
\end{frame}

\section{Introduction}

\begin{frame}{What is Quantum Error Correction?}
  \begin{itemize}
    \item Quantum computers are susceptible to errors from decoherence and noise
    \item Quantum error correction codes protect quantum information
    \item Key question: \textbf{How do we evaluate whether a quantum
      error correction code is good?}
      \begin{itemize}
        \item Can it correct a certain number of errors?
        \item What is the maximum number of correctable errors?
      \end{itemize}
  \end{itemize}
\end{frame}

\begin{frame}{What is the Surface Code?}
  \begin{itemize}
    \item One of the most promising quantum error correction codes
    \item Based on topological properties
    \item Qubits arranged in a 2D lattice
    \item Error correction performed through stabilizer measurements
    \item Widely studied and implemented in quantum computing platforms
  \end{itemize}
\end{frame}

\section{Encoding}

\begin{frame}{SAT Encoding Techniques}
  \textbf{Goal:} Encode quantum error correction verification as a SAT problem

  \vspace{1em}

  \textbf{Key Encodings:}
  \begin{itemize}
    \item \textbf{Totalizer encoding} for cardinality constraints
      \begin{itemize}
        \item Efficiently encode constraints like ``at most $k$ errors occur''
      \end{itemize}
    \item \textbf{Tree encoding} for XOR constraints
      \begin{itemize}
        \item Handle parity checks efficiently
      \end{itemize}
  \end{itemize}
\end{frame}

\begin{frame}{Verification Strategy}
  \textbf{Two types of problems:}

  \vspace{1em}

  \begin{block}{Can the code correct $k$ errors? (UNSAT Problem)}
    \begin{itemize}
      \item Try to find a counterexample with $\leq k$ errors that
        cannot be corrected
      \item If UNSAT, the code can correct $k$ errors
    \end{itemize}
  \end{block}

  \vspace{0.5em}

  \begin{block}{Can the code fail with $k$ errors? (SAT Problem)}
    \begin{itemize}
      \item Try to find an example with $\leq k$ errors that leads to failure
      \item If SAT, the code cannot correct all $k$-error cases
    \end{itemize}
  \end{block}
\end{frame}

\section{Results}

\begin{frame}{Bug Discovery in Nature Paper}
  \textbf{Major Achievement:} We successfully identified and verified
  a bug in a recently published Nature paper!

  \vspace{1em}

  \begin{table}
    \centering
    \begin{tabular}{ccc}
      \toprule
      \textbf{Distance} & \textbf{Actual Correctable Errors} &
      \textbf{Claimed} \\
      \midrule
      3  & 0 & 0 \\
      5  & 1 & 1 \\
      7  & 2 & 2 \\
      9  & 3 & 3 \\
      \rowcolor{red!20}
      11 & \textcolor{red}{\textbf{3}} & 4 \\
      \rowcolor{red!20}
      13 & \textcolor{red}{\textbf{4}} & 5 \\
      \bottomrule
    \end{tabular}
  \end{table}

  \vspace{0.5em}
  \small{The code fails to correct the claimed number of errors for
  distances 11 and 13!}
\end{frame}

\begin{frame}{Performance Results}
  \textbf{Bug Detection: Pretty Fast! $\checkmark$}

  \vspace{1em}

  \begin{itemize}
    \item SAT solver quickly finds counterexamples
    \item Verification of bug completed in reasonable time
    \item Demonstrates effectiveness of SAT-based approach
  \end{itemize}

  \vspace{1em}

  \begin{center}
    \textit{Runtime details: [Include specific timing results]}
  \end{center}
\end{frame}

\section{Challenges}

\begin{frame}{The Verification Challenge}
  \begin{block}{The Problem}
    We can propose a fix for the bug, but we \textbf{cannot verify}
    whether it works using SAT solvers alone.
  \end{block}

  \vspace{1em}

  \textbf{Verification is very slow...}

  \vspace{0.5em}

  \begin{itemize}
    \item Proving correctness requires solving UNSAT problems
    \item Much harder than finding bugs (SAT problems)
  \end{itemize}

  \vspace{1em}

  \begin{center}
    \textit{Runtime details: }
  \end{center}
\end{frame}

\begin{frame}{Why is This So Hard?}
  \textbf{Combination of SAT solver weaknesses:}

  \vspace{1em}

  \begin{enumerate}
    \item \textbf{UNSAT problems}
      \begin{itemize}
        \item Proving non-existence is inherently harder than finding examples
      \end{itemize}

    \item \textbf{XOR encodings}
      \begin{itemize}
        \item SAT solvers struggle with parity constraints
      \end{itemize}

    \item \textbf{Cardinality constraints}
      \begin{itemize}
        \item ``At most $k$ errors'' constraints are challenging
      \end{itemize}

    \item \textbf{Pigeonhole principle}
      \begin{itemize}
        \item Notoriously hard for SAT solvers
      \end{itemize}
  \end{enumerate}

  \vspace{0.5em}
  \small{Each alone is challenging; together they are formidable!}
\end{frame}

\section{Conclusion and Future Work}

\begin{frame}{A Hybrid Approach}
  \textbf{Leverage the strengths of different tools:}

  \vspace{1em}

  \begin{block}{SAT Solvers: Fast Pruning}
    \begin{itemize}
      \item Quickly find bugs and counterexamples
      \item Prune the search space efficiently
      \item Identify promising candidates
    \end{itemize}
  \end{block}

  \vspace{0.5em}

  \begin{block}{Lean Theorem Prover: Formal Verification}
    \begin{itemize}
      \item Formally verify correctness of proposed fixes
      \item Provide mathematical proof of error correction properties
      \item Guarantee correctness where SAT solvers struggle
    \end{itemize}
  \end{block}

  \vspace{1em}
  \centering
  \textbf{Combine SAT and Lean for comprehensive verification!}
\end{frame}

\begin{frame}{Summary}
  \begin{itemize}
    \item \textbf{Problem:} Verifying quantum error correction codes
    \item \textbf{Approach:} SAT solver with specialized encodings
    \item \textbf{Success:} Found bugs in published Nature paper
    \item \textbf{Challenge:} Verifying fixes is computationally hard
    \item \textbf{Future:} Hybrid SAT + Lean approach
  \end{itemize}

  \vspace{2em}

  \begin{center}
    \Large{\textbf{Thank you!}}

    \vspace{1em}

    \normalsize{Questions?}
  \end{center}
\end{frame}

\end{document}

